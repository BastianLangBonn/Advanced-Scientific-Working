%%%%%%%%%%%%%%%%%%%%%%%%%%%%%%%%%%%%%%%%%
% Beamer Presentation
% LaTeX Template
% Version 1.0 (10/11/12)
%
% This template has been downloaded from:
% http://www.LaTeXTemplates.com
%
% License:
% CC BY-NC-SA 3.0 (http://creativecommons.org/licenses/by-nc-sa/3.0/)
%
%%%%%%%%%%%%%%%%%%%%%%%%%%%%%%%%%%%%%%%%%

%----------------------------------------------------------------------------------------
%	PACKAGES AND THEMES
%----------------------------------------------------------------------------------------

\documentclass[8pt]{beamer}

\mode<presentation> {

% The Beamer class comes with a number of default slide themes
% which change the colors and layouts of slides. Below this is a list
% of all the themes, uncomment each in turn to see what they look like.

%\usetheme{default}
%\usetheme{AnnArbor}
%\usetheme{Antibes}
%\usetheme{Bergen}
\usetheme{Berkeley}
%\usetheme{Berlin}
%\usetheme{Boadilla}
%\usetheme{CambridgeUS}
%\usetheme{Copenhagen}
%\usetheme{Darmstadt}
%\usetheme{Dresden}
%\usetheme{Frankfurt}
%\usetheme{Goettingen}
%\usetheme{Hannover}
%\usetheme{Ilmenau}
%\usetheme{JuanLesPins}
%\usetheme{Luebeck}
%\usetheme{Madrid}
%\usetheme{Malmoe}
%\usetheme{Marburg}
%\usetheme{Montpellier}
%\usetheme{PaloAlto}
%\usetheme{Pittsburgh}
%\usetheme{Rochester}
%\usetheme{Singapore}
%\usetheme{Szeged}
%\usetheme{Warsaw}

% As well as themes, the Beamer class has a number of color themes
% for any slide theme. Uncomment each of these in turn to see how it
% changes the colors of your current slide theme.

%\usecolortheme{albatross}
%\usecolortheme{beaver}
%\usecolortheme{beetle}
%\usecolortheme{crane}
%\usecolortheme{dolphin}
%\usecolortheme{dove}
%\usecolortheme{fly}
%\usecolortheme{lily}
%\usecolortheme{orchid}
%\usecolortheme{rose}
%\usecolortheme{seagull}
%\usecolortheme{seahorse}
%\usecolortheme{whale}
%\usecolortheme{wolverine}

%\setbeamertemplate{footline} % To remove the footer line in all slides uncomment this line
\setbeamertemplate{footline}[page number] % To replace the footer line in all slides with a simple slide count uncomment this line

%\setbeamertemplate{navigation symbols}{} % To remove the navigation symbols from the bottom of all slides uncomment this line
}

\usepackage{graphicx} % Allows including images
\usepackage{booktabs} % Allows the use of \toprule, \midrule and \bottomrule in tables
\usepackage{tabularx}  % for 'tabularx' environment and 'X' column type
\usepackage{ragged2e}  % for '\RaggedRight' macro (allows hyphenation)
\newcolumntype{Y}{>{\RaggedRight\arraybackslash}X} 

\setcounter{figure}{0}

%----------------------------------------------------------------------------------------
%	TITLE PAGE
%----------------------------------------------------------------------------------------

\title[R \& D]{R \& D\\About the Reader} % The short title appears at the bottom of every slide, the full title is only on the title page
\author{Bastian Lang} % Your name
\institute[BRSU] % Your institution as it will appear on the bottom of every slide, may be shorthand to save space
{
Master of Autonomous Systems \\ % Your institution for the title page
}
\date{April 6, 2015} 

\begin{document}


\listoffigures
\begin{frame}
\titlepage 
\end{frame}

%----------------------------------------------------------------------------------------
%	PRESENTATION SLIDES
%----------------------------------------------------------------------------------------

\begin{frame}
	\frametitle{Reader Profile}
	Who will be the typical reader of the report?
	\begin{itemize}
		\item Researchers/students in the field of computer science, evolutionary robotics and energy efficient vehicle control
	\end{itemize}
	Who should be able to understand at least half of it?
	\begin{itemize}
		\item Anyone that understand the high level concepts of evolutionary algorithms, artificial neural networks and that has a good mathematical background
		\item Anyone with the ability to understand and follow abstract concepts, algorithms and ideas.
	\end{itemize}
\end{frame}

\begin{frame}
	\frametitle{Reader Profile}
	What knowledge do you expect the reader to have on a...
	\begin{itemize}
		\item General level
		\begin{itemize}
			\item Basic functionality of an artificial neural network
			\item ANNs can be used for control tasks
			\item EAs can be used to design ANNs
			\item There are algorithms to apply EAs to ANNs
			\item literature: Haykin, Simon, and Neural Network. "A comprehensive foundation." Neural Networks 2.2004 (2004).
		\end{itemize}
		
		\item More Specific Level
		\begin{itemize}
			\item Weights and the topology determine the functionality an ANN represents
			\item EAs can evolve the weights and the topology of an ANN w.r.t. a given task
			\item literature: Haykin, Simon, and Neural Network. "A comprehensive foundation." Neural Networks 2.2004 (2004).
		\end{itemize}
		\item Detailed Level
		\begin{itemize}
			\item -none-
			\item literature: -none specific-
		\end{itemize}
	\end{itemize}
\end{frame}

\begin{frame}
\frametitle{What shall the reader gain/learn?\\General Level}
\begin{itemize}
	\item SOA approaches use Graph Search 
	\item ANNs are able to outperform Graph Search approaches
	\item Transferring solutions onto real devices is non-trivial
	\item ANNs can be evolved for optimal energy efficient control
	\item Evolved ANNs have not been used on real devices for energy efficient vehicle control so far
\end{itemize}
\end{frame}

\begin{frame}
\frametitle{What shall the reader gain/learn?\\Specific Level}
\begin{itemize}
	\item Graph Search approaches have memory problems
	\item ANNs do not need much storage
	\item Problem with reality gap stems from inaccurate simulations
	\item Accurate simulations need too much computation time and memory
\end{itemize}
\end{frame}

\begin{frame}
\frametitle{What shall the reader gain/learn?\\Detailed Level}
\begin{itemize}
	\item The higher the level of discretization the more memory Graph Search does need
	\item ANNs are do not require a previous discretization of the problem
	\item NEAT is an SOA approach that can be used to design an ANN for energy efficient vehicle control
	\item The reality gap can shrink using more elaborate motor models
	\item The transferability approach can minimize the reality gap
\end{itemize}
\end{frame}

\begin{frame}
	\frametitle{Estimation of Work Spend on Different Problems}
	\begin{itemize}
		\item Current SOA and deficiencies: ca 3 pages
		\begin{itemize}
			\item Using Graph Search for Vehicle Control
			\item Neuroevolution
			\item Reality Gap
		\end{itemize}
		\item Explaining used algorithms and concepts
		\begin{itemize}
			\item NEAT: ca 2-3p
			\item Reality Gap: ca 2-3p
		\end{itemize}				 
	\end{itemize}
\end{frame}


\end{document} 