%%%%%%%%%%%%%%%%%%%%%%%%%%%%%%%%%%%%%%%%%
% Short Sectioned Assignment
% LaTeX Template
% Version 1.0 (5/5/12)
%
% This template has been downloaded from:
% http://www.LaTeXTemplates.com
%
% Original author:
% Frits Wenneker (http://www.howtotex.com)
%
% License:
% CC BY-NC-SA 3.0 (http://creativecommons.org/licenses/by-nc-sa/3.0/)
%
%%%%%%%%%%%%%%%%%%%%%%%%%%%%%%%%%%%%%%%%%

%----------------------------------------------------------------------------------------
%	PACKAGES AND OTHER DOCUMENT CONFIGURATIONS
%----------------------------------------------------------------------------------------

\documentclass[paper=a4, fontsize=11pt]{scrartcl} % A4 paper and 11pt font size

\usepackage[T1]{fontenc} % Use 8-bit encoding that has 256 glyphs
\usepackage{fourier} % Use the Adobe Utopia font for the document - comment this line to return to the LaTeX default
\usepackage[english]{babel} % English language/hyphenation
\usepackage{amsmath,amsfonts,amsthm} % Math packages

\usepackage{graphicx}

\usepackage{sectsty} % Allows customizing section commands
\allsectionsfont{\centering \normalfont\scshape} % Make all sections centered, the default font and small caps

\usepackage{fancyhdr} % Custom headers and footers
\pagestyle{fancyplain} % Makes all pages in the document conform to the custom headers and footers
\fancyhead{} % No page header - if you want one, create it in the same way as the footers below
\fancyfoot[L]{} % Empty left footer
\fancyfoot[C]{} % Empty center footer
\fancyfoot[R]{\thepage} % Page numbering for right footer
\renewcommand{\headrulewidth}{0pt} % Remove header underlines
\renewcommand{\footrulewidth}{0pt} % Remove footer underlines
\setlength{\headheight}{13.6pt} % Customize the height of the header

\numberwithin{equation}{section} % Number equations within sections (i.e. 1.1, 1.2, 2.1, 2.2 instead of 1, 2, 3, 4)
\numberwithin{figure}{section} % Number figures within sections (i.e. 1.1, 1.2, 2.1, 2.2 instead of 1, 2, 3, 4)
\numberwithin{table}{section} % Number tables within sections (i.e. 1.1, 1.2, 2.1, 2.2 instead of 1, 2, 3, 4)

\setlength\parindent{0pt} % Removes all indentation from paragraphs - comment this line for an assignment with lots of text

%----------------------------------------------------------------------------------------
%	TITLE SECTION
%----------------------------------------------------------------------------------------

\newcommand{\horrule}[1]{\rule{\linewidth}{#1}} % Create horizontal rule command with 1 argument of height

\title{	
\normalfont \normalsize 
\textsc{BRSU} \\ [25pt] % Your university, school and/or department name(s)
\horrule{0.5pt} \\[0.4cm] % Thin top horizontal rule
\huge Advanced Scientific Working\\-Essay-\\
Energy efficient control of rail vehicles % The assignment title
\horrule{2pt} \\[0.5cm] % Thick bottom horizontal rule
}

\author{Bastian Lang} % Your name

\date{\normalsize\today} % Today's date or a custom date

\begin{document}

\maketitle % Print the title

\newpage

\section{Reference}
Golovitcher, Iakov M. "Energy efficient control of rail vehicles." Systems, Man, and Cybernetics, 2001 IEEE International Conference on. Vol. 1. IEEE, 2001.

\section{Abstract}

This paper describes an analytical method of computation of optimal controls which minimize the energy consumption by rail or any other fixed path vehicles. A specific aspect of this problem is that the external forces applied to the vehicle and the maximum allowable speed depend on the coordinate of the vehicle. The known analytical solutions


\section{Essay}
\subsection{What is the paper about?}
\begin{itemize}
	\item Analytical method for computation of optimal controls
	\item Minimize energy consumption for rail other fixed path vehicles
	
\end{itemize}

\subsection{Why is this relevant?}
\begin{itemize}
	\item Raising energy prices
	\item Environmental concerns
\end{itemize}

\subsection{What have others done and why is this not sufficient?}
\begin{itemize}
	\item Classical numerical methods of optimization
	\begin{itemize}
		\item Significant computation time
		\item No real time calculations possible
	\end{itemize}
	\item Applying analytical methods of optimal control theory
	\begin{itemize}
		\item Simplified assumptions about tracks
	\end{itemize}
	\item Set of optimal controls for a short section of track
	\begin{itemize}
		\item Ignoring some constraints on velocity
	\end{itemize}
\end{itemize}

\subsection{What have the author's done and why is this better?}
\begin{itemize}
	\item Analytical solution of optimization problem
	\item No simplifications - Including steep climbs and descents
	\item Apply the maximum principle
	\item Use control-switching graphs to find a sequence of controls
\end{itemize}



\subsection{How did they evaluate their solution?}
\begin{itemize}
	\item Computer simulation of calculated strategy
	\item Application of strategy on several known tracks and schedules
	\item Computed savings of 3\% for subway lines not having any spare time
	\item Optimizing timetables for a long segment for local, intercity and freight traffic resulted in about 7\% savings
\end{itemize}

\subsection{Scientific Deficit}
\begin{itemize}
	\item Computations not during runtime
	\item System parameters are assumed to be known apriori
	\begin{itemize}
		\item Gao, Shigen, et al. "Approximation-based robust adaptive automatic train control: an approach for actuator saturation." Intelligent Transportation Systems, IEEE Transactions on 14.4 (2013): 1733-1742.
		\item Shigen, Gao, et al. "Characteristic model-based golden section adaptive control for high-speed train." Control Conference (CCC), 2012 31st Chinese. IEEE, 2012.
	\end{itemize}
	\item Not suitable for high speed lines
	\begin{itemize}
		\item Sicre, Carlos, et al. "Modeling and optimizing energy‐efficient manual driving on high‐speed lines." IEEJ Transactions on Electrical and Electronic Engineering 7.6 (2012): 633-640.
		\item Sicre, C., A. P. Cucala, and Antonio Fernandez-Cardador. "Real time regulation of efficient driving of high speed trains based on a genetic algorithm and a fuzzy model of manual driving." Engineering Applications of Artificial Intelligence 29 (2014): 79-92.
	\end{itemize}

\end{itemize}

\subsection{Scientific Contribution}
\begin{itemize}
	\item Very efficient approach
	\begin{itemize}
		\item Wang, Yihui, et al. "A survey on optimal trajectory planning for train operations." Service Operations, Logistics, and Informatics (SOLI), 2011 IEEE International Conference on. IEEE, 2011.
	\end{itemize}
	\item Applicable for short distance lines
	\begin{itemize}
		\item Sicre, C., A. P. Cucala, and Antonio Fernandez-Cardador. "Real time regulation of efficient driving of high speed trains based on a genetic algorithm and a fuzzy model of manual driving." Engineering Applications of Artificial Intelligence 29 (2014): 79-92.
		\item Sicre, Carlos, et al. "Modeling and optimizing energy‐efficient manual driving on high‐speed lines." IEEJ Transactions on Electrical and Electronic Engineering 7.6 (2012): 633-640.
	\end{itemize}
\end{itemize}
\end{document}