%%%%%%%%%%%%%%%%%%%%%%%%%%%%%%%%%%%%%%%%%
% Beamer Presentation
% LaTeX Template
% Version 1.0 (10/11/12)
%
% This template has been downloaded from:
% http://www.LaTeXTemplates.com
%
% License:
% CC BY-NC-SA 3.0 (http://creativecommons.org/licenses/by-nc-sa/3.0/)
%
%%%%%%%%%%%%%%%%%%%%%%%%%%%%%%%%%%%%%%%%%

%----------------------------------------------------------------------------------------
%	PACKAGES AND THEMES
%----------------------------------------------------------------------------------------

\documentclass[8pt]{beamer}

\mode<presentation> {

% The Beamer class comes with a number of default slide themes
% which change the colors and layouts of slides. Below this is a list
% of all the themes, uncomment each in turn to see what they look like.

%\usetheme{default}
%\usetheme{AnnArbor}
%\usetheme{Antibes}
%\usetheme{Bergen}
\usetheme{Berkeley}
%\usetheme{Berlin}
%\usetheme{Boadilla}
%\usetheme{CambridgeUS}
%\usetheme{Copenhagen}
%\usetheme{Darmstadt}
%\usetheme{Dresden}
%\usetheme{Frankfurt}
%\usetheme{Goettingen}
%\usetheme{Hannover}
%\usetheme{Ilmenau}
%\usetheme{JuanLesPins}
%\usetheme{Luebeck}
%\usetheme{Madrid}
%\usetheme{Malmoe}
%\usetheme{Marburg}
%\usetheme{Montpellier}
%\usetheme{PaloAlto}
%\usetheme{Pittsburgh}
%\usetheme{Rochester}
%\usetheme{Singapore}
%\usetheme{Szeged}
%\usetheme{Warsaw}

% As well as themes, the Beamer class has a number of color themes
% for any slide theme. Uncomment each of these in turn to see how it
% changes the colors of your current slide theme.

%\usecolortheme{albatross}
%\usecolortheme{beaver}
%\usecolortheme{beetle}
%\usecolortheme{crane}
%\usecolortheme{dolphin}
%\usecolortheme{dove}
%\usecolortheme{fly}
%\usecolortheme{lily}
%\usecolortheme{orchid}
%\usecolortheme{rose}
%\usecolortheme{seagull}
%\usecolortheme{seahorse}
%\usecolortheme{whale}
%\usecolortheme{wolverine}

%\setbeamertemplate{footline} % To remove the footer line in all slides uncomment this line
\setbeamertemplate{footline}[page number] % To replace the footer line in all slides with a simple slide count uncomment this line

%\setbeamertemplate{navigation symbols}{} % To remove the navigation symbols from the bottom of all slides uncomment this line
}

\usepackage{graphicx} % Allows including images
\usepackage{booktabs} % Allows the use of \toprule, \midrule and \bottomrule in tables
\usepackage{tabularx}  % for 'tabularx' environment and 'X' column type
\usepackage{ragged2e}  % for '\RaggedRight' macro (allows hyphenation)
\newcolumntype{Y}{>{\RaggedRight\arraybackslash}X} 

\setcounter{figure}{0}

%----------------------------------------------------------------------------------------
%	TITLE PAGE
%----------------------------------------------------------------------------------------

\title[R \& D]{R \& D\\Second two Ws} % The short title appears at the bottom of every slide, the full title is only on the title page
\author{Bastian Lang} % Your name
\institute[BRSU] % Your institution as it will appear on the bottom of every slide, may be shorthand to save space
{
Master of Autonomous Systems \\ % Your institution for the title page
}
\date{April 6, 2015} 

\begin{document}


\listoffigures
%\begin{frame}
%\titlepage 
%\end{frame}

%----------------------------------------------------------------------------------------
%	PRESENTATION SLIDES
%----------------------------------------------------------------------------------------


\begin{frame}
	\frametitle{Practical design of minimal energy controls for an electric bicycle}
	\textit{Grossoleil, David, and Dominique Meizel. "Practical design of minimal energy controls for an electric bicycle." 9th International Conference on Modeling, Optimization \& SIMulation. 2012.}\vspace{5mm}
	
	Approach
	\begin{itemize}
		\item Sample search space for discretization
		\item Build search graph
		\item Use A* to find minimal cost path
		\item Use heuristic based on friction, air drag, kinetic energy and potential energy
	\end{itemize}
	
	Deficits
	\begin{itemize}
		\item Graph search approach requires discretization of search space
		\item The higher the rate of discretization, the more complex (especially space)
	\end{itemize}
	
	Contributions
	\begin{itemize}
		\item Drastically reduced the search space for graph search approach
	\end{itemize}
\end{frame}

\begin{frame}
	\frametitle{Explicit fuel optimal speed profiles for heavy trucks on a set of topographic road profiles}
	\textit{Froeberg, Anders, Erik Hellstroem, and Lars Nielsen. Explicit fuel optimal speed profiles for heavy trucks on a set of topographic road profiles. No. 2006-01-1071. SAE Technical Paper, 2006.}\vspace{5mm}
	
	Approach
	\begin{itemize}
		\item Usage of a physical model to derive efficient driving behaviour
		\item Take model situations compute the optimal fuel supply
		\begin{itemize}
			\item level road
			\item small gradients
			\item high uphill slopes
			\item high downhill slopes
		\end{itemize}
	\end{itemize}
	
	Deficit
	\begin{itemize}
		\item Not applicable to unknown tracks
	\end{itemize}
	
	Contribution
	\begin{itemize}
		\item Optimal control can be achieved using only three different motor controls
		\begin{itemize}
			\item Roll
			\item Speed
			\item Cruise
		\end{itemize}
	\end{itemize}
\end{frame}


\begin{frame}
	\frametitle{Evolving Look Ahead Controllers for Energy
Optimal Driving and Path Planning}
\textit{Gaier, Adam, and Alexander Asteroth. "Evolving look ahead controllers for energy optimal driving and path planning." Innovations in Intelligent Systems and Applications (INISTA) Proceedings, 2014 IEEE International Symposium on. IEEE, 2014.}\vspace{5mm}

Approach
\begin{itemize}
	\item Use evolved neural network for energy efficient vehicle control in simulation
	\item Use simple, but continuous model
	\item Use only three commands: Roll, Speed, Cruise
	\item Use SOA approach NEAT
	\begin{itemize}
		\item Optimization of topology and weights of a neural network
		\item Starts with minimal topologies and complexifies them
		\item Uses special form of cross-over operator
		\item Promotes novel solutions
	\end{itemize}
	
\end{itemize}
Deficits
\begin{itemize}
	\item Proof of concept evaluated in simulation $\rightarrow$ Probably has to be modified to be usable in reality (Reality Gap)
	\begin{itemize}
		\item Use more detailed models of vehicle and environment
		\item Use transferability approach
	\end{itemize}
\end{itemize}	

Contributions
\begin{itemize}
	\item Controller that is adaptable and has a very low space complexity
\end{itemize}
	
\end{frame}




\end{document} 