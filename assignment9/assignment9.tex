%%%%%%%%%%%%%%%%%%%%%%%%%%%%%%%%%%%%%%%%%
% Beamer Presentation
% LaTeX Template
% Version 1.0 (10/11/12)
%
% This template has been downloaded from:
% http://www.LaTeXTemplates.com
%
% License:
% CC BY-NC-SA 3.0 (http://creativecommons.org/licenses/by-nc-sa/3.0/)
%
%%%%%%%%%%%%%%%%%%%%%%%%%%%%%%%%%%%%%%%%%

%----------------------------------------------------------------------------------------
%	PACKAGES AND THEMES
%----------------------------------------------------------------------------------------

\documentclass[8pt]{beamer}

\mode<presentation> {

% The Beamer class comes with a number of default slide themes
% which change the colors and layouts of slides. Below this is a list
% of all the themes, uncomment each in turn to see what they look like.

%\usetheme{default}
%\usetheme{AnnArbor}
%\usetheme{Antibes}
%\usetheme{Bergen}
\usetheme{Berkeley}
%\usetheme{Berlin}
%\usetheme{Boadilla}
%\usetheme{CambridgeUS}
%\usetheme{Copenhagen}
%\usetheme{Darmstadt}
%\usetheme{Dresden}
%\usetheme{Frankfurt}
%\usetheme{Goettingen}
%\usetheme{Hannover}
%\usetheme{Ilmenau}
%\usetheme{JuanLesPins}
%\usetheme{Luebeck}
%\usetheme{Madrid}
%\usetheme{Malmoe}
%\usetheme{Marburg}
%\usetheme{Montpellier}
%\usetheme{PaloAlto}
%\usetheme{Pittsburgh}
%\usetheme{Rochester}
%\usetheme{Singapore}
%\usetheme{Szeged}
%\usetheme{Warsaw}

% As well as themes, the Beamer class has a number of color themes
% for any slide theme. Uncomment each of these in turn to see how it
% changes the colors of your current slide theme.

%\usecolortheme{albatross}
%\usecolortheme{beaver}
%\usecolortheme{beetle}
%\usecolortheme{crane}
%\usecolortheme{dolphin}
%\usecolortheme{dove}
%\usecolortheme{fly}
%\usecolortheme{lily}
%\usecolortheme{orchid}
%\usecolortheme{rose}
%\usecolortheme{seagull}
%\usecolortheme{seahorse}
%\usecolortheme{whale}
%\usecolortheme{wolverine}

%\setbeamertemplate{footline} % To remove the footer line in all slides uncomment this line
\setbeamertemplate{footline}[page number] % To replace the footer line in all slides with a simple slide count uncomment this line

%\setbeamertemplate{navigation symbols}{} % To remove the navigation symbols from the bottom of all slides uncomment this line
}

\usepackage{graphicx} % Allows including images
\usepackage{booktabs} % Allows the use of \toprule, \midrule and \bottomrule in tables
\usepackage{tabularx}  % for 'tabularx' environment and 'X' column type
\usepackage{ragged2e}  % for '\RaggedRight' macro (allows hyphenation)
\newcolumntype{Y}{>{\RaggedRight\arraybackslash}X} 

\setcounter{figure}{0}

%----------------------------------------------------------------------------------------
%	TITLE PAGE
%----------------------------------------------------------------------------------------

\title[Assignment 9]{ASW\\Assignment 9} % The short title appears at the bottom of every slide, the full title is only on the title page
\author{Bastian Lang} % Your name
\institute[BRSU] % Your institution as it will appear on the bottom of every slide, may be shorthand to save space
{
Master of Autonomous Systems \\ % Your institution for the title page
}
\date{April 6, 2015} 

\begin{document}


\listoffigures
%\begin{frame}
%\titlepage 
%\end{frame}

%----------------------------------------------------------------------------------------
%	PRESENTATION SLIDES
%----------------------------------------------------------------------------------------


\begin{frame}
	\frametitle{Criteria for reviewing a paper}
	\begin{itemize}
		\item Do the authors make clear what the paper is about in the beginning?
		\item Reasons given why related work is not sufficient?
		\item Approach explained properly?
		\item Sufficient experiments that supports their intent?
		\begin{itemize}
			\item Compared to other algorithms?
			\item Special cases?
			\item ...
		\end{itemize}
		\item Conclusions based on experiments?
		\item Style of writing clear?
	\end{itemize}

\end{frame}

\begin{frame}
	\frametitle{Evolving neural networks through augmenting topologies.}
	\textit{Stanley, Kenneth O., and Risto Miikkulainen. "Evolving neural networks through augmenting topologies." Evolutionary computation 10.2 (2002): 99-127.}\vspace{5mm}
	Likes:
	\begin{itemize}
		\item Detailed and extensive evaluation
		\begin{itemize}
			\item Analysis of importance of each of introduced mechanisms
		\end{itemize}
		\item Well explained and impressive approach
		\begin{itemize}
			\item Good justification of each mechanism
		\end{itemize}
	\end{itemize}
	

\end{frame}

\begin{frame}
	\frametitle{The transferability approach: Crossing the reality gap in evolutionary robotics.}
	\textit{Koos, Sylvain, Jean-Baptiste Mouret, and Stéphane Doncieux. "The transferability approach: Crossing the reality gap in evolutionary robotics." Evolutionary Computation, IEEE Transactions on 17.1 (2013): 122-145.}\vspace{5mm}
	Likes:
	\begin{itemize}
		\item Sophisticated analysis of the results
		\begin{itemize}
			\item Reasons for difference of different approaches
		\end{itemize}
		\item Good presentations of results
		\begin{itemize}
			\item Many visualizing figures 
		\end{itemize}
		\item Good explanation of approach
		\item Extensive state of the art discussion
	\end{itemize}
	
\end{frame}

\begin{frame}
	\frametitle{Evolving Look Ahead Controllers for Energy
Optimal Driving and Path Planning}
\textit{Gaier, Adam, and Alexander Asteroth. "Evolving look ahead controllers for energy optimal driving and path planning." Innovations in Intelligent Systems and Applications (INISTA) Proceedings, 2014 IEEE International Symposium on. IEEE, 2014.}\vspace{5mm}

	Likes:
	\begin{itemize}
		\item Good state of the art discussion
		\begin{itemize}
			\item deficits/contributions of other approaches
		\end{itemize}
		\item Well explained approach
		\begin{itemize}
			\item Description of all involved approaches
		\end{itemize}
		\item Logical composition of paper
		\item Thorough analysis of results
	\end{itemize}


	
\end{frame}


\end{document} 