%%%%%%%%%%%%%%%%%%%%%%%%%%%%%%%%%%%%%%%%%
% Short Sectioned Assignment
% LaTeX Template
% Version 1.0 (5/5/12)
%
% This template has been downloaded from:
% http://www.LaTeXTemplates.com
%
% Original author:
% Frits Wenneker (http://www.howtotex.com)
%
% License:
% CC BY-NC-SA 3.0 (http://creativecommons.org/licenses/by-nc-sa/3.0/)
%
%%%%%%%%%%%%%%%%%%%%%%%%%%%%%%%%%%%%%%%%%

%----------------------------------------------------------------------------------------
%	PACKAGES AND OTHER DOCUMENT CONFIGURATIONS
%----------------------------------------------------------------------------------------

\documentclass[paper=a4, fontsize=11pt]{scrartcl} % A4 paper and 11pt font size

\usepackage[T1]{fontenc} % Use 8-bit encoding that has 256 glyphs
\usepackage{fourier} % Use the Adobe Utopia font for the document - comment this line to return to the LaTeX default
\usepackage[english]{babel} % English language/hyphenation
\usepackage{amsmath,amsfonts,amsthm} % Math packages

\usepackage{graphicx}

\usepackage{sectsty} % Allows customizing section commands
\allsectionsfont{\centering \normalfont\scshape} % Make all sections centered, the default font and small caps

\usepackage{fancyhdr} % Custom headers and footers
\pagestyle{fancyplain} % Makes all pages in the document conform to the custom headers and footers
\fancyhead{} % No page header - if you want one, create it in the same way as the footers below
\fancyfoot[L]{} % Empty left footer
\fancyfoot[C]{} % Empty center footer
\fancyfoot[R]{\thepage} % Page numbering for right footer
\renewcommand{\headrulewidth}{0pt} % Remove header underlines
\renewcommand{\footrulewidth}{0pt} % Remove footer underlines
\setlength{\headheight}{13.6pt} % Customize the height of the header

\numberwithin{equation}{section} % Number equations within sections (i.e. 1.1, 1.2, 2.1, 2.2 instead of 1, 2, 3, 4)
\numberwithin{figure}{section} % Number figures within sections (i.e. 1.1, 1.2, 2.1, 2.2 instead of 1, 2, 3, 4)
\numberwithin{table}{section} % Number tables within sections (i.e. 1.1, 1.2, 2.1, 2.2 instead of 1, 2, 3, 4)

\setlength\parindent{0pt} % Removes all indentation from paragraphs - comment this line for an assignment with lots of text

%----------------------------------------------------------------------------------------
%	TITLE SECTION
%----------------------------------------------------------------------------------------

\newcommand{\horrule}[1]{\rule{\linewidth}{#1}} % Create horizontal rule command with 1 argument of height

\title{	
\normalfont \normalsize 
\textsc{BRSU} \\ [25pt] % Your university, school and/or department name(s)
\horrule{0.5pt} \\[0.4cm] % Thin top horizontal rule
\huge Advanced Scientific Working\\-Reading Report-\\
Evolving Look Ahead Controllers for Energy Optimal Driving and Path Planning % The assignment title
\horrule{2pt} \\[0.5cm] % Thick bottom horizontal rule
}

\author{Bastian Lang} % Your name

\date{\normalsize\today} % Today's date or a custom date

\begin{document}

\maketitle % Print the title

\newpage

\section{Reference}
Gaier, A., \& Asteroth, A. (2014, June). Evolving look ahead controllers for energy optimal driving and path planning. In Innovations in Intelligent Systems and Applications (INISTA) Proceedings, 2014 IEEE International Symposium on (pp. 138-145). IEEE.

\section{Reading Report}
\subsection{Problem Formulation}
\begin{itemize}
	\item Developing techniques for near energy-optimal control of vehicles using controllers that base their decisions on characteristics of the road.
	\item Developing control strategies that make the best use of the terrain.
\end{itemize}

\subsection{Why is this still a problem?}
\begin{itemize}
	\item Engineering of transport systems resulted in efficient designs, but further optimizations are diminishing.
	\item Driver-training programs have yielded impressive results.
	\item Instead of being reduced, emissions caused by transportation drastically increased in 2005 and 2007.
\end{itemize}

\subsection{State of the art}
\begin{itemize}
	\item Energy-optimal control of commercial trucks and trains
	\item Energy-optimal control of experimental vehicles
	\item Energy-optimal control of hybrid gas-electric vehicles
	\item Finding fuel-optimal behavior of an experimental fuel-cell vehicle using optimal control theory and a simplified model of environment and vehicle
	\item It was found that only three commands are needed to produce an optimal strategy:
	\begin{itemize}
		\item full power
		\item maintain velocity
		\item coast
	\end{itemize}
	\item Reformulating the problem as a graph-search problem using A* and a special heuristic the number of nodes needed could be drastically decreased with still achieving near-optimal solutions
	\item Using dynamic programming to reduce complexity and rounding errors from state space discretisation
	\item Using inverted system equations real valued controls could be used. But not all equations can be inverted and it is often non-trivial
	\item Avoiding numerical issues by first defining the boundary line between feasible and infeasible states. This reduced computation costs by an order of magnitude.
	\item Neuroevolution techniques have shown to be more effective on some benchmark tests than reinforcement learning techniques
	\item NEAT is the current most successful algorithm within neuroevolutionary algorithms
\end{itemize}

\subsection{What is the author's contribution?}
\begin{itemize}
	\item Use of evolutionary algorithms instead of graph search algorithms
	\item Application of a state-of-the-art evolutionary algorithm to the energy-optimal control problem
\end{itemize}

\subsection{How did the solution solve the addressed problem?}
\begin{itemize}
	\item By using the method of NEAT (NeuroEvolution of Augmented Topologies) the authors evolve a controller for an e-bike 
	\item Use of simplified vehicle model 
	\item Use of three motor commands
	\item Evolutionary algorithms do not need to work in a discretized state space, so rounding errors are no problem
	\item Evolutionary algorithms do not use state space models. Therefore increasing the complexity of the model does not have a huge effect on the space complexity.
	\item Using a local solution that calculates new commands at every meter, a deggree of precision can be achieved that is not possible for graph search in most scenarios.
\end{itemize}


\subsection{Evaluation}
\begin{itemize}
	\item Comparison of performance on 35 routes with a Graph Search algorithm
	\item Training of net using cross-validation
	\item Training on 5 maps, testing on remaining 30
	\item 50 training runs, each containing of evolving a population of size 150 for 1000 generations.
	\item Comparison to graph search in three different resolution levels
	\item Performance is only slightly worse than the highest depth graph.
	\item Analysed in detail most solutions performed better
	\item Computation time was nearly instantaneous versus about 1 minute for an 8km route
	\item Space complexity was about 1KB versus 11GB
\end{itemize}

\subsection{Scientific Deficit}
\begin{itemize}
	\item The approach taken uses a very popular algorithm from evolutionary algorithms, but it does not use any techniques to ensure that the evolved controller will be able to run on a real device. Where the authors can give a prove of concept that an evolutionary algorithm may be better suited than a graph search approach, like this it probably is not usable in practise. Although giving prove of concept was the author's goal and this is not a scientific deficit but a shortcoming of this approach so far. 
	\begin{itemize}
		\item Miglino, O., Lund, H. H., \& Nolfi, S. (1995). Evolving mobile robots in simulated and real environments. Artificial life, 2(4), 417-434.
	\item Jakobi, N., Husbands, P., \& Harvey, I. (1995). Noise and the reality gap: The use of simulation in evolutionary robotics. In Advances in artificial life (pp. 704-720). Springer Berlin Heidelberg.
	\item Koos, S., Mouret, J. B., \& Doncieux, S. (2013). The transferability approach: Crossing the reality gap in evolutionary robotics. Evolutionary Computation, IEEE Transactions on, 17(1), 122-145.
	\end{itemize}
\end{itemize}

\subsection{Scientific Contribution}
\begin{itemize}
	\item The authors were able to show that using evolutionary algorithms for optimizing energy-efficient driving outperforms graph search algorithms in terms of computation time and space needed and produces solutions that are nearly as optimal as solutions from very detailed graph search algorithms.
	\item As for this paper is very recent (2014) I am not able to quote any other work that supports my statement. Citeseer does not know this paper yet and google just gives one quote from a paper of the same authors.
\end{itemize}
\end{document}